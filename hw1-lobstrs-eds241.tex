% Options for packages loaded elsewhere
\PassOptionsToPackage{unicode}{hyperref}
\PassOptionsToPackage{hyphens}{url}
%
\documentclass[
]{article}
\usepackage{amsmath,amssymb}
\usepackage{iftex}
\ifPDFTeX
  \usepackage[T1]{fontenc}
  \usepackage[utf8]{inputenc}
  \usepackage{textcomp} % provide euro and other symbols
\else % if luatex or xetex
  \usepackage{unicode-math} % this also loads fontspec
  \defaultfontfeatures{Scale=MatchLowercase}
  \defaultfontfeatures[\rmfamily]{Ligatures=TeX,Scale=1}
\fi
\usepackage{lmodern}
\ifPDFTeX\else
  % xetex/luatex font selection
\fi
% Use upquote if available, for straight quotes in verbatim environments
\IfFileExists{upquote.sty}{\usepackage{upquote}}{}
\IfFileExists{microtype.sty}{% use microtype if available
  \usepackage[]{microtype}
  \UseMicrotypeSet[protrusion]{basicmath} % disable protrusion for tt fonts
}{}
\makeatletter
\@ifundefined{KOMAClassName}{% if non-KOMA class
  \IfFileExists{parskip.sty}{%
    \usepackage{parskip}
  }{% else
    \setlength{\parindent}{0pt}
    \setlength{\parskip}{6pt plus 2pt minus 1pt}}
}{% if KOMA class
  \KOMAoptions{parskip=half}}
\makeatother
\usepackage{xcolor}
\usepackage[margin=1in]{geometry}
\usepackage{color}
\usepackage{fancyvrb}
\newcommand{\VerbBar}{|}
\newcommand{\VERB}{\Verb[commandchars=\\\{\}]}
\DefineVerbatimEnvironment{Highlighting}{Verbatim}{commandchars=\\\{\}}
% Add ',fontsize=\small' for more characters per line
\usepackage{framed}
\definecolor{shadecolor}{RGB}{248,248,248}
\newenvironment{Shaded}{\begin{snugshade}}{\end{snugshade}}
\newcommand{\AlertTok}[1]{\textcolor[rgb]{0.94,0.16,0.16}{#1}}
\newcommand{\AnnotationTok}[1]{\textcolor[rgb]{0.56,0.35,0.01}{\textbf{\textit{#1}}}}
\newcommand{\AttributeTok}[1]{\textcolor[rgb]{0.13,0.29,0.53}{#1}}
\newcommand{\BaseNTok}[1]{\textcolor[rgb]{0.00,0.00,0.81}{#1}}
\newcommand{\BuiltInTok}[1]{#1}
\newcommand{\CharTok}[1]{\textcolor[rgb]{0.31,0.60,0.02}{#1}}
\newcommand{\CommentTok}[1]{\textcolor[rgb]{0.56,0.35,0.01}{\textit{#1}}}
\newcommand{\CommentVarTok}[1]{\textcolor[rgb]{0.56,0.35,0.01}{\textbf{\textit{#1}}}}
\newcommand{\ConstantTok}[1]{\textcolor[rgb]{0.56,0.35,0.01}{#1}}
\newcommand{\ControlFlowTok}[1]{\textcolor[rgb]{0.13,0.29,0.53}{\textbf{#1}}}
\newcommand{\DataTypeTok}[1]{\textcolor[rgb]{0.13,0.29,0.53}{#1}}
\newcommand{\DecValTok}[1]{\textcolor[rgb]{0.00,0.00,0.81}{#1}}
\newcommand{\DocumentationTok}[1]{\textcolor[rgb]{0.56,0.35,0.01}{\textbf{\textit{#1}}}}
\newcommand{\ErrorTok}[1]{\textcolor[rgb]{0.64,0.00,0.00}{\textbf{#1}}}
\newcommand{\ExtensionTok}[1]{#1}
\newcommand{\FloatTok}[1]{\textcolor[rgb]{0.00,0.00,0.81}{#1}}
\newcommand{\FunctionTok}[1]{\textcolor[rgb]{0.13,0.29,0.53}{\textbf{#1}}}
\newcommand{\ImportTok}[1]{#1}
\newcommand{\InformationTok}[1]{\textcolor[rgb]{0.56,0.35,0.01}{\textbf{\textit{#1}}}}
\newcommand{\KeywordTok}[1]{\textcolor[rgb]{0.13,0.29,0.53}{\textbf{#1}}}
\newcommand{\NormalTok}[1]{#1}
\newcommand{\OperatorTok}[1]{\textcolor[rgb]{0.81,0.36,0.00}{\textbf{#1}}}
\newcommand{\OtherTok}[1]{\textcolor[rgb]{0.56,0.35,0.01}{#1}}
\newcommand{\PreprocessorTok}[1]{\textcolor[rgb]{0.56,0.35,0.01}{\textit{#1}}}
\newcommand{\RegionMarkerTok}[1]{#1}
\newcommand{\SpecialCharTok}[1]{\textcolor[rgb]{0.81,0.36,0.00}{\textbf{#1}}}
\newcommand{\SpecialStringTok}[1]{\textcolor[rgb]{0.31,0.60,0.02}{#1}}
\newcommand{\StringTok}[1]{\textcolor[rgb]{0.31,0.60,0.02}{#1}}
\newcommand{\VariableTok}[1]{\textcolor[rgb]{0.00,0.00,0.00}{#1}}
\newcommand{\VerbatimStringTok}[1]{\textcolor[rgb]{0.31,0.60,0.02}{#1}}
\newcommand{\WarningTok}[1]{\textcolor[rgb]{0.56,0.35,0.01}{\textbf{\textit{#1}}}}
\usepackage{graphicx}
\makeatletter
\def\maxwidth{\ifdim\Gin@nat@width>\linewidth\linewidth\else\Gin@nat@width\fi}
\def\maxheight{\ifdim\Gin@nat@height>\textheight\textheight\else\Gin@nat@height\fi}
\makeatother
% Scale images if necessary, so that they will not overflow the page
% margins by default, and it is still possible to overwrite the defaults
% using explicit options in \includegraphics[width, height, ...]{}
\setkeys{Gin}{width=\maxwidth,height=\maxheight,keepaspectratio}
% Set default figure placement to htbp
\makeatletter
\def\fps@figure{htbp}
\makeatother
\setlength{\emergencystretch}{3em} % prevent overfull lines
\providecommand{\tightlist}{%
  \setlength{\itemsep}{0pt}\setlength{\parskip}{0pt}}
\setcounter{secnumdepth}{-\maxdimen} % remove section numbering
\ifLuaTeX
  \usepackage{selnolig}  % disable illegal ligatures
\fi
\usepackage{bookmark}
\IfFileExists{xurl.sty}{\usepackage{xurl}}{} % add URL line breaks if available
\urlstyle{same}
\hypersetup{
  pdftitle={Assignment 1: California Spiny Lobster Abundance (Panulirus Interruptus)},
  pdfauthor={Eva Newby},
  hidelinks,
  pdfcreator={LaTeX via pandoc}}

\title{Assignment 1: California Spiny Lobster Abundance (\emph{Panulirus
Interruptus})}
\usepackage{etoolbox}
\makeatletter
\providecommand{\subtitle}[1]{% add subtitle to \maketitle
  \apptocmd{\@title}{\par {\large #1 \par}}{}{}
}
\makeatother
\subtitle{Assessing the Impact of Marine Protected Areas (MPAs) at 5
Reef Sites in Santa Barbara County}
\author{Eva Newby}
\date{1/8/2024 (Due 1/26)}

\begin{document}
\maketitle

\begin{center}\rule{0.5\linewidth}{0.5pt}\end{center}

\includegraphics{figures/spiny2.jpg}

\begin{center}\rule{0.5\linewidth}{0.5pt}\end{center}

\subsubsection{Assignment instructions:}\label{assignment-instructions}

\begin{itemize}
\item
  Working with partners to troubleshoot code and concepts is encouraged!
  If you work with a partner, please list their name next to yours at
  the top of your assignment so Annie and I can easily see who
  collaborated.
\item
  All written responses must be written independently (\textbf{in your
  own words}).
\item
  Please follow the question prompts carefully and include only the
  information each question asks in your submitted responses.
\item
  Submit both your knitted document and the associated
  \texttt{RMarkdown} or \texttt{Quarto} file.
\item
  Your knitted presentation should meet the quality you'd submit to
  research colleagues or feel confident sharing publicly. Refer to the
  rubric for details about presentation standards.
\end{itemize}

\textbf{Assignment submission (YOUR NAME):} Eva Newby
\_\_\_\_\_\_\_\_\_\_\_\_\_\_\_\_\_\_\_\_\_\_\_\_\_\_\_\_\_\_\_\_\_\_\_\_\_\_

\begin{center}\rule{0.5\linewidth}{0.5pt}\end{center}

\begin{Shaded}
\begin{Highlighting}[]
\CommentTok{\# Load libraries}
\FunctionTok{library}\NormalTok{(tidyverse)}
\FunctionTok{library}\NormalTok{(here)}
\FunctionTok{library}\NormalTok{(janitor)}
\FunctionTok{library}\NormalTok{(estimatr)  }
\FunctionTok{library}\NormalTok{(performance)}
\FunctionTok{library}\NormalTok{(jtools)}
\FunctionTok{library}\NormalTok{(gt)}
\FunctionTok{library}\NormalTok{(gtsummary)}
\FunctionTok{library}\NormalTok{(MASS) }\DocumentationTok{\#\# }\AlertTok{NOTE}\DocumentationTok{: The \textasciigrave{}select()\textasciigrave{} function is masked. Use: \textasciigrave{}dplyr::select()\textasciigrave{} \#\#}
\FunctionTok{library}\NormalTok{(interactions) }
\FunctionTok{library}\NormalTok{(ggridges)}
\FunctionTok{library}\NormalTok{(ggbeeswarm)}
\end{Highlighting}
\end{Shaded}

\begin{center}\rule{0.5\linewidth}{0.5pt}\end{center}

\paragraph{DATA SOURCE:}\label{data-source}

Reed D. 2019. SBC LTER: Reef: Abundance, size and fishing effort for
California Spiny Lobster (Panulirus interruptus), ongoing since 2012.
Environmental Data Initiative.
\url{https://doi.org/10.6073/pasta/a593a675d644fdefb736750b291579a0}.
Dataset accessed 11/17/2019.

\begin{center}\rule{0.5\linewidth}{0.5pt}\end{center}

\subsubsection{\texorpdfstring{\textbf{Introduction}}{Introduction}}\label{introduction}

You're about to dive into some deep data collected from five reef sites
in Santa Barbara County, all about the abundance of California spiny
lobsters! 🦞 Data was gathered by divers annually from 2012 to 2018
across Naples, Mohawk, Isla Vista, Carpinteria, and Arroyo Quemado
reefs.

Why lobsters? Well, this sample provides an opportunity to evaluate the
impact of Marine Protected Areas (MPAs) established on January 1, 2012
(Reed, 2019). Of these five reefs, Naples, and Isla Vista are MPAs,
while the other three are not protected (non-MPAs). Comparing lobster
health between these protected and non-protected areas gives us the
chance to study how commercial and recreational fishing might impact
these ecosystems.

We will consider the MPA sites the \texttt{treatment} group and use
regression methods to explore whether protecting these reefs really
makes a difference compared to non-MPA sites (our control group). In
this assignment, we'll think deeply about which causal inference
assumptions hold up under the research design and identify where they
fall short.

Let's break it down step by step and see what the data reveals! 📊

\includegraphics{figures/map-5reefs.png}

\begin{center}\rule{0.5\linewidth}{0.5pt}\end{center}

Step 1: Anticipating potential sources of selection bias

\textbf{a.} Do the control sites (Arroyo Quemado, Carpenteria, and
Mohawk) provide a strong counterfactual for our treatment sites (Naples,
Isla Vista)? Write a paragraph making a case for why this comparison is
centris paribus or whether selection bias is likely (be specific!).

The control sites may not provide a perfect counterfactual for the
treatment sites due to potential selection bias. For example, using the
centris paribus logic in this case means that the single difference
between the two groups is whether or not they are an MPA. However, there
may be other differences between the habitats, such as microclimate
and/or proximity to town/cities. One can make a case that centris
paribus logic may apply as the sites are relatively close together.

\begin{center}\rule{0.5\linewidth}{0.5pt}\end{center}

Step 2: Read \& wrangle data

\textbf{a.} Read in the raw data. Name the data.frame (\texttt{df})
\texttt{rawdata}

\textbf{b.} Use the function \texttt{clean\_names()} from the
\texttt{janitor} package

\begin{Shaded}
\begin{Highlighting}[]
\CommentTok{\# HINT: check for coding of missing values (\textasciigrave{}na = "{-}99999"\textasciigrave{})}

\NormalTok{rawdata }\OtherTok{\textless{}{-}} \FunctionTok{read\_csv}\NormalTok{(}\FunctionTok{here}\NormalTok{(}\StringTok{"data"}\NormalTok{, }\StringTok{\textquotesingle{}spiny\_abundance\_sb\_18.csv\textquotesingle{}}\NormalTok{), }\AttributeTok{na =} \StringTok{"{-}99999"}\NormalTok{) }\SpecialCharTok{\%\textgreater{}\%}
    \FunctionTok{clean\_names}\NormalTok{()}
\end{Highlighting}
\end{Shaded}

\textbf{c.} Create a new \texttt{df} named \texttt{tidyata}. Using the
variable \texttt{site} (reef location) create a new variable
\texttt{reef} as a \texttt{factor} and add the following labels in the
order listed (i.e., re-order the \texttt{levels}):

\begin{verbatim}
"Arroyo Quemado", "Carpenteria", "Mohawk", "Isla Vista",  "Naples"
\end{verbatim}

\begin{Shaded}
\begin{Highlighting}[]
\CommentTok{\# Add long lables to our sites and save in a col named reef}
\NormalTok{tidydata }\OtherTok{\textless{}{-}}\NormalTok{ rawdata }\SpecialCharTok{\%\textgreater{}\%} 
    \FunctionTok{mutate}\NormalTok{(}\AttributeTok{reef =} \FunctionTok{factor}\NormalTok{(site, }
                         \AttributeTok{levels =} \FunctionTok{c}\NormalTok{(}\StringTok{"AQUE"}\NormalTok{, }\StringTok{"CARP"}\NormalTok{, }\StringTok{"MOHK"}\NormalTok{,     }\StringTok{"IVEE"}\NormalTok{, }\StringTok{"NAPL"}\NormalTok{), }
                         \AttributeTok{labels =} \FunctionTok{c}\NormalTok{(}\StringTok{"Arroyo Quemado"}\NormalTok{, }\StringTok{"Carpenteria"}\NormalTok{, }\StringTok{"Mohawk"}\NormalTok{, }
                                    \StringTok{"Isla Vista"}\NormalTok{,  }\StringTok{"Naples"}\NormalTok{)))}
\end{Highlighting}
\end{Shaded}

Create new \texttt{df} named \texttt{spiny\_counts}

\textbf{d.} Create a new variable \texttt{counts} to allow for an
analysis of lobster counts where the unit-level of observation is the
total number of observed lobsters per \texttt{site}, \texttt{year} and
\texttt{transect}.

\begin{itemize}
\tightlist
\item
  Create a variable \texttt{mean\_size} from the variable
  \texttt{size\_mm}
\item
  NOTE: The variable \texttt{counts} should have values which are
  integers (whole numbers).
\item
  Make sure to account for missing cases (\texttt{na})!
\end{itemize}

\textbf{e.} Create a new variable \texttt{mpa} with levels \texttt{MPA}
and \texttt{non\_MPA}. For our regression analysis create a numerical
variable \texttt{treat} where MPA sites are coded \texttt{1} and
non\_MPA sites are coded \texttt{0}

\begin{Shaded}
\begin{Highlighting}[]
\CommentTok{\#HINT(d): Use \textasciigrave{}group\_by()\textasciigrave{} \& \textasciigrave{}summarize()\textasciigrave{} to provide the total number of lobsters observed at each site{-}year{-}transect row{-}observation. }

\CommentTok{\#HINT(e): Use \textasciigrave{}case\_when()\textasciigrave{} to create the 3 new variable columns}

\CommentTok{\# assign each site either mpa or non{-}mpa, and 1 or 0}
\NormalTok{spiny\_counts }\OtherTok{\textless{}{-}}\NormalTok{ tidydata }\SpecialCharTok{\%\textgreater{}\%} 
    \FunctionTok{group\_by}\NormalTok{(site, year, transect) }\SpecialCharTok{\%\textgreater{}\%} 
    \FunctionTok{summarise}\NormalTok{(}\AttributeTok{count =} \FunctionTok{sum}\NormalTok{(count, }\AttributeTok{na.rm =} \ConstantTok{TRUE}\NormalTok{), }\AttributeTok{mean\_size =} \FunctionTok{mean}\NormalTok{(size\_mm, }\AttributeTok{na.rm =} \ConstantTok{TRUE}\NormalTok{)) }\SpecialCharTok{\%\textgreater{}\%} 
    \FunctionTok{mutate}\NormalTok{(}\AttributeTok{mpa =} \FunctionTok{case\_when}\NormalTok{(site }\SpecialCharTok{\%in\%} \FunctionTok{c}\NormalTok{(}\StringTok{"IVEE"}\NormalTok{, }\StringTok{"NAPL"}\NormalTok{) }\SpecialCharTok{\textasciitilde{}} \StringTok{"MPA"}\NormalTok{,}
                           \AttributeTok{.default =} \StringTok{"non\_MPA"}\NormalTok{)) }\SpecialCharTok{\%\textgreater{}\%} 
    \FunctionTok{mutate}\NormalTok{(}\AttributeTok{treat =} \FunctionTok{case\_when}\NormalTok{(mpa }\SpecialCharTok{==} \StringTok{"MPA"} \SpecialCharTok{\textasciitilde{}} \DecValTok{1}\NormalTok{,}
                             \AttributeTok{.default =} \DecValTok{0}\NormalTok{)) }\SpecialCharTok{\%\textgreater{}\%} 
    \FunctionTok{ungroup}\NormalTok{()}
\end{Highlighting}
\end{Shaded}

\begin{quote}
NOTE: This step is crucial to the analysis. Check with a friend or come
to TA/instructor office hours to make sure the counts are coded
correctly!
\end{quote}

\begin{center}\rule{0.5\linewidth}{0.5pt}\end{center}

Step 3: Explore \& visualize data

\textbf{a.} Take a look at the data! Get familiar with the data in each
\texttt{df} format (\texttt{tidydata}, \texttt{spiny\_counts})

\begin{Shaded}
\begin{Highlighting}[]
\CommentTok{\# Data Exploration}
\CommentTok{\# First 5 rows of each}
\FunctionTok{print}\NormalTok{(}\FunctionTok{head}\NormalTok{(spiny\_counts))}
\FunctionTok{print}\NormalTok{(}\FunctionTok{head}\NormalTok{(tidydata))}

\CommentTok{\# Datatypes of each column}
\FunctionTok{print}\NormalTok{(spiny\_counts }\SpecialCharTok{\%\textgreater{}\%} \FunctionTok{summarise\_all}\NormalTok{(class))}
\FunctionTok{print}\NormalTok{(tidydata }\SpecialCharTok{\%\textgreater{}\%} \FunctionTok{summarise\_all}\NormalTok{(class))}
\end{Highlighting}
\end{Shaded}

\textbf{b.} We will focus on the variables \texttt{count},
\texttt{year}, \texttt{site}, and \texttt{treat}(\texttt{mpa}) to model
lobster abundance. Create the following 4 plots using a different method
each time from the 6 options provided. Add a layer (\texttt{geom}) to
each of the plots including informative descriptive statistics (you
choose; e.g., mean, median, SD, quartiles, range). Make sure each plot
dimension is clearly labeled (e.g., axes, groups).

\begin{itemize}
\tightlist
\item
  \href{https://r-charts.com/distribution/density-plot-group-ggplot2}{Density
  plot}
\item
  \href{https://r-charts.com/distribution/ggridges/}{Ridge plot}
\item
  \href{https://ggplot2.tidyverse.org/reference/geom_jitter.html}{Jitter
  plot}
\item
  \href{https://r-charts.com/distribution/violin-plot-group-ggplot2}{Violin
  plot}
\item
  \href{https://r-charts.com/distribution/histogram-density-ggplot2/}{Histogram}
\item
  \href{https://r-charts.com/distribution/beeswarm/}{Beeswarm}
\end{itemize}

Create plots displaying the distribution of lobster \textbf{counts}:

\begin{enumerate}
\def\labelenumi{\arabic{enumi})}
\tightlist
\item
  grouped by reef site
\item
  grouped by MPA status
\item
  grouped by year
\end{enumerate}

Create a plot of lobster \textbf{size} :

\begin{enumerate}
\def\labelenumi{\arabic{enumi})}
\setcounter{enumi}{3}
\tightlist
\item
  You choose the grouping variable(s)!
\end{enumerate}

\begin{Shaded}
\begin{Highlighting}[]
\CommentTok{\# plot 1: Beeswarm Plot grouped by MPA status}
\NormalTok{spiny\_counts }\SpecialCharTok{\%\textgreater{}\%} 
\FunctionTok{ggplot}\NormalTok{(}\FunctionTok{aes}\NormalTok{(}\AttributeTok{x =}\NormalTok{ treat, }\AttributeTok{y =}\NormalTok{ count, }\AttributeTok{color =}\NormalTok{ mpa)) }\SpecialCharTok{+}
    \FunctionTok{geom\_beeswarm}\NormalTok{(}\AttributeTok{alpha =} \FloatTok{0.5}\NormalTok{) }\SpecialCharTok{+}
    \FunctionTok{labs}\NormalTok{(}\AttributeTok{title =} \StringTok{\textquotesingle{}Beeswarm Plot of Lobster Count per MPA status\textquotesingle{}}\NormalTok{,}
         \AttributeTok{x =} \StringTok{\textquotesingle{}MPA status\textquotesingle{}}\NormalTok{,}
         \AttributeTok{y =} \StringTok{\textquotesingle{}Lobster Count\textquotesingle{}}\NormalTok{)}\SpecialCharTok{+}
    \FunctionTok{theme\_minimal}\NormalTok{()}\SpecialCharTok{+}
    \FunctionTok{geom\_boxplot}\NormalTok{(}\FunctionTok{aes}\NormalTok{(}\AttributeTok{x =}\NormalTok{ treat, }\AttributeTok{y =}\NormalTok{ count, }\AttributeTok{color =}\NormalTok{ mpa), }\AttributeTok{alpha =}\FloatTok{0.5}\NormalTok{) }\CommentTok{\# Statistical summary present in boxplot}
\end{Highlighting}
\end{Shaded}

\begin{Shaded}
\begin{Highlighting}[]
\CommentTok{\# plot 2: Ridge plot grouped by Reef}

\NormalTok{spiny\_counts }\SpecialCharTok{\%\textgreater{}\%} 
\FunctionTok{ggplot}\NormalTok{(}\FunctionTok{aes}\NormalTok{(}\AttributeTok{x =}\NormalTok{ count, }\AttributeTok{y =}\NormalTok{ site, }\AttributeTok{fill =}\NormalTok{ site))}\SpecialCharTok{+}
    \FunctionTok{geom\_density\_ridges}\NormalTok{(}\AttributeTok{alpha =} \FloatTok{0.8}\NormalTok{)}\SpecialCharTok{+}
     \FunctionTok{stat\_summary}\NormalTok{(}
        \AttributeTok{fun =}\NormalTok{ median, }
        \AttributeTok{geom =} \StringTok{"point"}\NormalTok{,}
        \FunctionTok{aes}\NormalTok{(}\AttributeTok{shape =} \StringTok{\textquotesingle{}Median\textquotesingle{}}\NormalTok{), }\CommentTok{\# Add in summary statistic, median}
        \AttributeTok{color =} \StringTok{"black"}\NormalTok{, }
        \AttributeTok{size =} \DecValTok{2}
\NormalTok{    )}\SpecialCharTok{+}
    \FunctionTok{theme\_minimal}\NormalTok{() }\SpecialCharTok{+} 
    \FunctionTok{theme}\NormalTok{(}\AttributeTok{axis.text.y =} \FunctionTok{element\_blank}\NormalTok{(),}
        \AttributeTok{axis.ticks.y =} \FunctionTok{element\_blank}\NormalTok{()) }\SpecialCharTok{+}
     \FunctionTok{labs}\NormalTok{(}\AttributeTok{title =} \StringTok{\textquotesingle{}Ridge Plot of Lobster Count per Reef\textquotesingle{}}\NormalTok{,}
         \AttributeTok{x =} \StringTok{\textquotesingle{}Lobster Count\textquotesingle{}}\NormalTok{,}
         \AttributeTok{y =} \StringTok{\textquotesingle{}Reef Site\textquotesingle{}}\NormalTok{)}
\end{Highlighting}
\end{Shaded}

\begin{Shaded}
\begin{Highlighting}[]
\CommentTok{\# Plot 3: Histogram grouped by year (help)}

\NormalTok{spiny\_counts }\SpecialCharTok{\%\textgreater{}\%} 
\FunctionTok{ggplot}\NormalTok{(}\FunctionTok{aes}\NormalTok{(}\AttributeTok{x =}\NormalTok{ year, }\AttributeTok{y =}\NormalTok{ count)) }\SpecialCharTok{+} 
    \FunctionTok{geom\_histogram}\NormalTok{(}\AttributeTok{stat =} \StringTok{\textquotesingle{}identity\textquotesingle{}}\NormalTok{,}
                   \AttributeTok{fill =} \StringTok{\textquotesingle{}cornflowerblue\textquotesingle{}}\NormalTok{, }
                   \AttributeTok{binwidth =} \DecValTok{1}\NormalTok{) }\SpecialCharTok{+}
    \FunctionTok{labs}\NormalTok{(}\AttributeTok{title =} \StringTok{"Histogram of Lobster Counts by Year"}\NormalTok{,}
         \AttributeTok{x =} \StringTok{"Year"}\NormalTok{,}
         \AttributeTok{y =} \StringTok{"Lobster Count"}\NormalTok{) }\SpecialCharTok{+}
    \FunctionTok{theme\_minimal}\NormalTok{() }\SpecialCharTok{+}
    \FunctionTok{stat\_summary}\NormalTok{(}\AttributeTok{fun =}\NormalTok{ spiny\_counts}\SpecialCharTok{$}\FunctionTok{count}\NormalTok{(mean), }
                 \AttributeTok{geom =} \StringTok{\textquotesingle{}vline\textquotesingle{}}\NormalTok{,}
\NormalTok{                 ) }\CommentTok{\# average count }

\CommentTok{\# Add summary statistic}
\end{Highlighting}
\end{Shaded}

\begin{Shaded}
\begin{Highlighting}[]
\CommentTok{\# Plot 4: Jitter plot grouped by lobster size}

\NormalTok{spiny\_counts }\SpecialCharTok{\%\textgreater{}\%} 
\FunctionTok{ggplot}\NormalTok{(}\FunctionTok{aes}\NormalTok{(}\AttributeTok{x =}\NormalTok{ site,}
           \AttributeTok{y =}\NormalTok{ mean\_size,}
           \AttributeTok{color =}\NormalTok{ site))}\SpecialCharTok{+}
    \FunctionTok{geom\_jitter}\NormalTok{(}\AttributeTok{alpha =} \FloatTok{0.7}\NormalTok{)}\SpecialCharTok{+}
    \FunctionTok{geom\_boxplot}\NormalTok{(}\AttributeTok{alpha =} \FloatTok{0.5}\NormalTok{) }\CommentTok{\# summary statistics present in boxplot}
\end{Highlighting}
\end{Shaded}

\textbf{c.} Compare means of the outcome by treatment group. Using the
\texttt{tbl\_summary()} function from the package
\href{https://www.danieldsjoberg.com/gtsummary/articles/tbl_summary.html}{\texttt{gt\_summary}}

\begin{Shaded}
\begin{Highlighting}[]
\CommentTok{\# USE: gt\_summary::tbl\_summary()}

\NormalTok{spiny\_counts }\SpecialCharTok{\%\textgreater{}\%}
\NormalTok{  dplyr}\SpecialCharTok{::}\FunctionTok{select}\NormalTok{(count, treat) }\SpecialCharTok{\%\textgreater{}\%}
  \FunctionTok{tbl\_summary}\NormalTok{(}
    \AttributeTok{by =}\NormalTok{ treat,  }\CommentTok{\# Group by MPA treatment status (0 or 1)}
    \AttributeTok{statistic =} \FunctionTok{list}\NormalTok{(}\FunctionTok{all\_continuous}\NormalTok{() }\SpecialCharTok{\textasciitilde{}} \StringTok{"\{mean\}"}\NormalTok{),  }\CommentTok{\# Show mean }
    \AttributeTok{missing =} \StringTok{"no"}  \CommentTok{\# remove missing values}
\NormalTok{  ) }\SpecialCharTok{\%\textgreater{}\%}
  \FunctionTok{add\_p}\NormalTok{(}\AttributeTok{test =} \FunctionTok{list}\NormalTok{(count }\SpecialCharTok{\textasciitilde{}} \StringTok{"t.test"}\NormalTok{))  }\CommentTok{\# Perform t{-}test to compare means}
\end{Highlighting}
\end{Shaded}

\begin{center}\rule{0.5\linewidth}{0.5pt}\end{center}

Step 4: OLS regression- building intuition

\textbf{a.} Start with a simple OLS estimator of lobster counts
regressed on treatment. Use the function \texttt{summ()} from the
\href{https://jtools.jacob-long.com/}{\texttt{jtools}} package to print
the OLS output

\textbf{b.} Interpret the intercept \& predictor coefficients \emph{in
your own words}. Use full sentences and write your interpretation of the
regression results to be as clear as possible to a non-academic
audience.

\begin{Shaded}
\begin{Highlighting}[]
\CommentTok{\# }\AlertTok{NOTE}\CommentTok{: We will not evaluate/interpret model fit in this assignment (e.g., R{-}square)}

\NormalTok{m1\_ols }\OtherTok{\textless{}{-}} \FunctionTok{lm}\NormalTok{(count }\SpecialCharTok{\textasciitilde{}}\NormalTok{ treat, spiny\_counts)}

\FunctionTok{summ}\NormalTok{(m1\_ols, }\AttributeTok{model.fit =} \ConstantTok{FALSE}\NormalTok{) }
\end{Highlighting}
\end{Shaded}

Interpretation: On average, there are 22.73 lobsters counted in non-MPA
sites (intercept, or Beta 0) whereas MPA sites have 5.36 more lobsters
on average compared to non-MPA sites (Beta 1). However, this difference
could be due to chance (as our p-value is 0.3, not statistically
significant).

\textbf{c.} Check the model assumptions using the \texttt{check\_model}
function from the \texttt{performance} package

\textbf{d.} Explain the results of the 4 diagnostic plots. Why are we
getting this result?

\begin{Shaded}
\begin{Highlighting}[]
\FunctionTok{check\_model}\NormalTok{(m1\_ols,  }\AttributeTok{check =} \StringTok{"qq"}\NormalTok{ )}
\end{Highlighting}
\end{Shaded}

Explanation: The deviations from the lines suggest the data is not
normally distributed.

\begin{Shaded}
\begin{Highlighting}[]
\FunctionTok{check\_model}\NormalTok{(m1\_ols, }\AttributeTok{check =} \StringTok{"normality"}\NormalTok{)}
\end{Highlighting}
\end{Shaded}

Explanation: The skew suggests that the data is not normally
distributed.

\begin{Shaded}
\begin{Highlighting}[]
\FunctionTok{check\_model}\NormalTok{(m1\_ols, }\AttributeTok{check =} \StringTok{"homogeneity"}\NormalTok{)}
\end{Highlighting}
\end{Shaded}

Explanation: The graph shows a more U-shaped curve which indicates that
the variance of residuals is not constant across fitted values
(heteroscedasticity), and this violates one of the assumptions of OLS.
Additionally, the residuals are more extreme at either end, suggesting
that the model predictions are less accurate for extreme values.

\begin{Shaded}
\begin{Highlighting}[]
\FunctionTok{check\_model}\NormalTok{(m1\_ols, }\AttributeTok{check =} \StringTok{"pp\_check"}\NormalTok{)}
\end{Highlighting}
\end{Shaded}

Explanation: As our observed data doesn't seem to match our
model-predicted data that well, alternative models should be explored.

\begin{center}\rule{0.5\linewidth}{0.5pt}\end{center}

Step 5: Fitting GLMs

\textbf{a.} Estimate a Poisson regression model using the \texttt{glm()}
function

\textbf{b.} Interpret the predictor coefficient in your own words. Use
full sentences and write your interpretation of the results to be as
clear as possible to a non-academic audience.

\begin{Shaded}
\begin{Highlighting}[]
\CommentTok{\# Exponentiate the coefficients for more clear interpretation}
\FunctionTok{exp}\NormalTok{(}\FunctionTok{coef}\NormalTok{(m2\_pois))}

\CommentTok{\# Percentage change}
\NormalTok{(}\FloatTok{1.235956} \SpecialCharTok{{-}} \DecValTok{1}\NormalTok{) }\SpecialCharTok{*} \DecValTok{100}
\end{Highlighting}
\end{Shaded}

On average, the expected lobster count in non-MPA areas is 22.729
lobsters. On average, the expected lobster count in MPA areas is 1.235
times greater than in non-MPA areas, meaning MPA areas have an approx
23.6\% higher lobster count than non-MPA areas.

\textbf{c.} Explain the statistical concept of dispersion and
overdispersion in the context of this model.

Dispersion refers to how far away data is from the mean. An assumption
of the poisson model is that the variance and the mean are equal, and
that the data is not overdispersed (variance greater than the mean). If
the variance is greater than the mean, then a negative binomial
regression should be used instead of poisson.

\textbf{d.} Compare results with previous model, explain change in the
significance of the treatment effect

It is clear that OLS is not the best model for our data, as our data
violates several of OLS' assumptions (such as normal distribution). The
poisson model assesses a log-linear relationship, which appears to be
more accurate for the data. However, depending on whether or not there
is overdispersion, the standard errors in the poisson model may be
inaccurate. This could make it seem that the change in significance was
larger than it actually was.

\begin{Shaded}
\begin{Highlighting}[]
\CommentTok{\#HINT1: Incidence Ratio Rate (IRR): Exponentiation of beta returns coefficient which is interpreted as the \textquotesingle{}percent change\textquotesingle{} for a one unit increase in the predictor }

\CommentTok{\#HINT2: For the second glm() argument \textasciigrave{}family\textasciigrave{} use the following specification option \textasciigrave{}family = poisson(link = "log")\textasciigrave{}}

\NormalTok{m2\_pois }\OtherTok{\textless{}{-}} \FunctionTok{glm}\NormalTok{(count }\SpecialCharTok{\textasciitilde{}}\NormalTok{ treat, spiny\_counts, }\AttributeTok{family =} \FunctionTok{poisson}\NormalTok{(}\AttributeTok{link =} \StringTok{\textquotesingle{}log\textquotesingle{}}\NormalTok{))}

\FunctionTok{summary}\NormalTok{(m2\_pois)}
\end{Highlighting}
\end{Shaded}

\textbf{e.} Check the model assumptions. Explain results.

The key model assumptions for poisson is that the daya must represent
count data, the mean equals the variance (no overdispersion), there is a
log-linear relationship, zero-inflation (no zero counts in the data),
and observations must be independent of each other.

\textbf{f.} Conduct tests for over-dispersion \& zero-inflation. Explain
results.

\begin{Shaded}
\begin{Highlighting}[]
\FunctionTok{check\_model}\NormalTok{(m2\_pois)}
\end{Highlighting}
\end{Shaded}

\begin{Shaded}
\begin{Highlighting}[]
\FunctionTok{check\_overdispersion}\NormalTok{(m2\_pois)}
\end{Highlighting}
\end{Shaded}

\begin{Shaded}
\begin{Highlighting}[]
\FunctionTok{check\_zeroinflation}\NormalTok{(m2\_pois)}
\end{Highlighting}
\end{Shaded}

\textbf{g.} Fit a negative binomial model using the function glm.nb()
from the package \texttt{MASS} and check model diagnostics

\textbf{h.} In 1-2 sentences explain rationale for fitting this GLM
model.

As overdispersion was found to be present in the last step, a negative
binomial regression model is a better option for a more accurate fitting
compared to a poisson model.

\textbf{i.} Interpret the treatment estimate result in your own words.
Compare with results from the previous model.

This model appears to be a better fit, as overdispersion is not
occurring; however, the model is still over-fitting zeros.

\begin{Shaded}
\begin{Highlighting}[]
\CommentTok{\# }\AlertTok{NOTE}\CommentTok{: The \textasciigrave{}glm.nb()\textasciigrave{} function does not require a \textasciigrave{}family\textasciigrave{} argument}

\NormalTok{m3\_nb }\OtherTok{\textless{}{-}} \FunctionTok{glm.nb}\NormalTok{(count }\SpecialCharTok{\textasciitilde{}}\NormalTok{ treat, spiny\_counts)}

\FunctionTok{summary}\NormalTok{(m3\_nb)}
\end{Highlighting}
\end{Shaded}

\begin{Shaded}
\begin{Highlighting}[]
\FunctionTok{check\_overdispersion}\NormalTok{(m3\_nb)}
\end{Highlighting}
\end{Shaded}

\begin{Shaded}
\begin{Highlighting}[]
\FunctionTok{check\_zeroinflation}\NormalTok{(m3\_nb)}
\end{Highlighting}
\end{Shaded}

\begin{Shaded}
\begin{Highlighting}[]
\FunctionTok{check\_predictions}\NormalTok{(m3\_nb)}
\end{Highlighting}
\end{Shaded}

\begin{Shaded}
\begin{Highlighting}[]
\FunctionTok{check\_model}\NormalTok{(m3\_nb)}
\end{Highlighting}
\end{Shaded}

\begin{center}\rule{0.5\linewidth}{0.5pt}\end{center}

Step 6: Compare models

\textbf{a.} Use the \texttt{export\_summ()} function from the
\texttt{jtools} package to look at the three regression models you fit
side-by-side.

\textbf{c.} Write a short paragraph comparing the results. Is the
treatment effect \texttt{robust} or stable across the model
specifications.

There is variation in the treatment effect between the three models. The
results of the OLS model show no statistically significant effect on
lobster count in MPA vs non-MPA, where as both poisson and negative
binomial regression both show statistically significant effects. The
treatment effect is robust and/or stable across both the poisson and
negative binomial regression analysis, but not OLS.

\begin{Shaded}
\begin{Highlighting}[]
\CommentTok{\# Position 3 models side by side}
\FunctionTok{export\_summs}\NormalTok{(}\FunctionTok{list}\NormalTok{(m1\_ols, m2\_pois, m3\_nb),}
             \AttributeTok{model.names =} \FunctionTok{c}\NormalTok{(}\StringTok{"OLS"}\NormalTok{,}\StringTok{"Poisson"}\NormalTok{, }\StringTok{"NB"}\NormalTok{),}
             \AttributeTok{statistics =} \StringTok{"none"}\NormalTok{)}
\end{Highlighting}
\end{Shaded}

\begin{center}\rule{0.5\linewidth}{0.5pt}\end{center}

Step 7: Building intuition - fixed effects

\textbf{a.} Create new \texttt{df} with the \texttt{year} variable
converted to a factor

\textbf{b.} Run the following negative binomial model using
\texttt{glm.nb()}

\begin{itemize}
\item
  Add fixed effects for \texttt{year} (i.e., dummy coefficients)
\item
  Include an interaction term between variables \texttt{treat} \&
  \texttt{year} (\texttt{treat*year})
\end{itemize}

\textbf{c.} Take a look at the regression output. Each coefficient
provides a comparison or the difference in means for a specific
sub-group in the data. Informally, describe the what the model has
estimated at a conceptual level (NOTE: you do not have to interpret
coefficients individually)

The model has estimated that the interaction between treatment and count
vary depending on the year. Some years show that there is a strong
positive correlation between MPA/non-MPA and count, whereas other years
don't show that same pattern.

\textbf{d.} Explain why the main effect for treatment is negative? *Does
this result make sense?

This makes sense as there is still variation between treat and count per
year. The first two years observed (2012 and 2013) show negative
effects, however, they switch to positive for every year after that.
This also makes sense, as the effects from MPA implementation, such as
lack of lobster hunting, take a few years for lobster populations to
bounce back and see the effects in the counts.

\begin{Shaded}
\begin{Highlighting}[]
\NormalTok{ff\_counts }\OtherTok{\textless{}{-}}\NormalTok{ spiny\_counts }\SpecialCharTok{\%\textgreater{}\%} 
    \FunctionTok{mutate}\NormalTok{(}\AttributeTok{year=}\FunctionTok{as\_factor}\NormalTok{(year))}
    
\NormalTok{m5\_fixedeffs }\OtherTok{\textless{}{-}} \FunctionTok{glm.nb}\NormalTok{(}
\NormalTok{    count }\SpecialCharTok{\textasciitilde{}} 
\NormalTok{        treat }\SpecialCharTok{+}
\NormalTok{        year }\SpecialCharTok{+}
\NormalTok{        treat}\SpecialCharTok{*}\NormalTok{year,}
    \AttributeTok{data =}\NormalTok{ ff\_counts)}

\FunctionTok{summ}\NormalTok{(m5\_fixedeffs, }\AttributeTok{model.fit =} \ConstantTok{FALSE}\NormalTok{)}
\end{Highlighting}
\end{Shaded}

\textbf{e.} Look at the model predictions: Use the
\texttt{interact\_plot()} function from package \texttt{interactions} to
plot mean predictions by year and treatment status.

\textbf{f.} Re-evaluate your responses (c) and (b) above.

Based on the \texttt{interact\_plot()}, it is easier to determine the
differences in counts per year based on treat (MPA vs non-MPA). It
appears that the largest difference is in 2018, with MPA receiving much
higher counts than non-MPA. This makes sense looking at the results for
the previous step, \texttt{treat:year2018} received the highest number,
with 2.62, compared to all the other years.

\begin{Shaded}
\begin{Highlighting}[]
\FunctionTok{interact\_plot}\NormalTok{(m5\_fixedeffs, }\AttributeTok{pred =}\NormalTok{ year, }\AttributeTok{modx =}\NormalTok{ treat,}
              \AttributeTok{outcome.scale =} \StringTok{"response"}\NormalTok{) }\CommentTok{\# }\AlertTok{NOTE}\CommentTok{: \textquotesingle{}link\textquotesingle{} = y{-}axis on log{-}scale}

\CommentTok{\# HINT: Change \textasciigrave{}outcome.scale\textasciigrave{} to "response" to convert y{-}axis scale to counts}
\end{Highlighting}
\end{Shaded}

\textbf{g.} Using \texttt{ggplot()} create a plot in same style as the
previous \texttt{interaction\ plot}, but displaying the original scale
of the outcome variable (lobster counts). This type of plot is commonly
used to show how the treatment effect changes across discrete time
points (i.e., panel data).

The plot should have\ldots{} - \texttt{year} on the x-axis -
\texttt{counts} on the y-axis - \texttt{mpa} as the grouping variable

\begin{Shaded}
\begin{Highlighting}[]
\CommentTok{\# Hint 1: Group counts by \textasciigrave{}year\textasciigrave{} and \textasciigrave{}mpa\textasciigrave{} and calculate the \textasciigrave{}mean\_count\textasciigrave{}}

\NormalTok{plot\_counts }\OtherTok{\textless{}{-}}\NormalTok{ spiny\_counts }\SpecialCharTok{\%\textgreater{}\%} 
    \FunctionTok{group\_by}\NormalTok{(year, mpa) }\SpecialCharTok{\%\textgreater{}\%} 
    \FunctionTok{summarize}\NormalTok{(}\AttributeTok{mean\_count =} \FunctionTok{mean}\NormalTok{(count, }\AttributeTok{na.rm =} \ConstantTok{TRUE}\NormalTok{))}

\CommentTok{\# Hint 2: Convert variable \textasciigrave{}year\textasciigrave{} to a factor}
\NormalTok{plot\_counts}\SpecialCharTok{$}\NormalTok{year }\OtherTok{\textless{}{-}} \FunctionTok{as.factor}\NormalTok{(plot\_counts}\SpecialCharTok{$}\NormalTok{year)}
    
\CommentTok{\# plot Lobster counts per year and MPA status}
\NormalTok{plot\_counts }\SpecialCharTok{\%\textgreater{}\%} 
   \FunctionTok{ggplot}\NormalTok{(}\FunctionTok{aes}\NormalTok{(}\AttributeTok{x =}\NormalTok{ year, }
              \AttributeTok{y =}\NormalTok{ mean\_count, }
              \AttributeTok{color =}\NormalTok{ mpa, }
              \AttributeTok{group =}\NormalTok{ mpa)) }\SpecialCharTok{+} 
    \FunctionTok{geom\_line}\NormalTok{()}\SpecialCharTok{+}
    \FunctionTok{geom\_point}\NormalTok{()}\SpecialCharTok{+}
    \FunctionTok{labs}\NormalTok{(}\AttributeTok{x =} \StringTok{"Year"}\NormalTok{, }
       \AttributeTok{y =} \StringTok{"Lobster Counts"}\NormalTok{, }
       \AttributeTok{title =} \StringTok{"Lobster Counts per Year and MPA Status"}\NormalTok{,}
       \AttributeTok{color =} \StringTok{"MPA Status"}\NormalTok{)}\SpecialCharTok{+}
    \FunctionTok{theme\_minimal}\NormalTok{()}\SpecialCharTok{+}
    \FunctionTok{scale\_color\_manual}\NormalTok{(}\AttributeTok{values =} \FunctionTok{c}\NormalTok{(}\StringTok{"MPA"} \OtherTok{=} \StringTok{"\#F07167"}\NormalTok{, }\StringTok{"non\_MPA"} \OtherTok{=} \StringTok{"\#7284A8"}\NormalTok{))  }
\end{Highlighting}
\end{Shaded}

\begin{center}\rule{0.5\linewidth}{0.5pt}\end{center}

Step 8: Reconsider causal identification assumptions

\begin{enumerate}
\def\labelenumi{\alph{enumi}.}
\item
  Discuss whether you think \texttt{spillover\ effects} are likely in
  this research context (see Glossary of terms;
  \url{https://docs.google.com/document/d/1RIudsVcYhWGpqC-Uftk9UTz3PIq6stVyEpT44EPNgpE/edit?usp=sharing})

  Spillover effects suggest that ``one unit's treatment affects a
  control unit's outcome''. Given that there are location separations
  between the MPA and the non-MPA groups, and that the existence of MPAs
  within a certain area generally boost the biodiversity of the local
  ecosystem, I think it is fair to consider that there may be spillover
  effects. Lobsters don't know which areas are MPAs and which areas are
  not, and can roam freely between the two. This could potentially
  affect the lobster counts.
\item
  Explain why spillover is an issue for the identification of causal
  effects

  Spillover effects complicate identifying the true effect of the
  treatment (in this example, MPA vs non-MPA). This makes it more
  difficult to identify causality for certain effects.
\item
  How does spillover relate to impact in this research setting?'

  As mentioned in part a, the existence of MPAs may boost the general
  biodiversity of an area, even in non-MPA areas. This would affect the
  count amounts in both MPA and non-MPA areas.
\item
  Discuss the following causal inference assumptions in the context of
  the MPA treatment effect estimator. Evaluate if each of the assumption
  are reasonable:

  \begin{enumerate}
  \def\labelenumii{\arabic{enumii})}
  \tightlist
  \item
    SUTVA: Stable Unit Treatment Value assumption - states that each
    entity can only be affected by the treatment it is receiving , and
    not the treatment of others. Lobster counts are affected by many
    other factors than MPA vs non-MPA status, such as food availability
    and disease. The SUTVA assumption is not valid for this case.
  \item
    Excludability assumption - states that a variable used to create
    variation in the treatment affects the outcome only through its
    effect on the treatment, and not have any direct effect on the
    outcome. For this case, this assumption is reasonable as MPA itself
    doesn't directly affect lobster counts, it likely provides shelter
    from hunting, which may directly affect lobster counts.
  \end{enumerate}
\end{enumerate}

\begin{center}\rule{0.5\linewidth}{0.5pt}\end{center}

\section{EXTRA CREDIT}\label{extra-credit}

\begin{quote}
Use the recent lobster abundance data with observations collected up
until 2024 (\texttt{lobster\_sbchannel\_24.csv}) to run an analysis
evaluating the effect of MPA status on lobster counts using the same
focal variables.
\end{quote}

\begin{enumerate}
\def\labelenumi{\alph{enumi}.}
\tightlist
\item
  Create a new script for the analysis on the updated data
\item
  Run at least 3 regression models \& assess model diagnostics
\item
  Compare and contrast results with the analysis from the 2012-2018 data
  sample (\textasciitilde{} 2 paragraphs)
\end{enumerate}

\begin{center}\rule{0.5\linewidth}{0.5pt}\end{center}

\includegraphics{figures/spiny1.png}

\end{document}
